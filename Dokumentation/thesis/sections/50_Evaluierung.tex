\chapter[Evaluierung]{Evaluierung \small Oleg Geier}\label{ch:eval}

Auf die integrierte Physiks Engine von cocos2d haben wir verzichtet da unser Spiel nur von der Kollision zum Boden Gebrauch macht. Eine normale Physik Engine bezieht dafür noch viel mehr Parameter (bsp. Masse, Momentum, Gravitation) mit ein, was bei uns nicht gegeben ist oder mit einem unnötig hohen Mehraufwand verbunden.

Hinsichtlich der Performance haben wir uns auch wenig Gedanken gemacht. Hier bleibt noch offen Tests anzulegen um bestimmte Kriterien einzuhalten. Oder selbst einfache Tests zwischen zwei Funnktionen bsp. ist es effektiver viele kleine Bilder nach Bedarf zu laden oder ein großes Bild ständig im Speicher zu halten. Ist die implementierte Kollisionsabfrage effektiv?

Die recht langen Ladezeiten beim Starten eines Levels sollten mit einem Lade-Bildschirm überdeckt werden, damit für den Endanwender ersichtlich ist, dass etwas im Gange ist.

Als Verbesserung des Ablaufs bzw. Aufbaus der Animationen hätte man die Spawn Klasse verwenden können. Dadurch wäre eine variablere Verwendung von Sequenzen und Aktionen möglich. Diese hätte besonders bei den Angriffspattern der Boss Klasse Anwendung gefunden. Dadurch ist die komplette Boss Klasse momentan auf einen einzigen Endgegner fixiert.
