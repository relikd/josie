\chapter{Einleitung}\label{ch:intro}

\section{Ausgangssituation}

In der heutigen Zeit spielt man Videospiele nicht nur auf Computern oder Konsolen, sondern auch auf Mobiltelefonen. Der Markt von Spielen für mobile Geräte ist in den letzten Jahren rapide gewachsen und erfreut sich immer größerer Beliebtheit. Einer Studie von Bitcom aus dem Jahr 2014 zufolge, in der die beliebtesten Spieleplattformn ermittelt wurden, führt das Smartphone bzw. Handy mit 78\% um 9\% gegenüber dem stationären PC und ist somit an erster Stelle. Dieselbe Studie hat sich auch mit den beliebtesten und meist gespielten Spielegenres beschäftigt. Stratiegie- und Denkspiele sind laut Bitcom am beliebtesten, gefolgt von Gelegenheitsspielen, Actionspielen, Social Games, Jump n' Runs und Renn- und Sportspielen.




\section{Motivation}

Im bisherigen Verlauf unseres Informatik-Studiums hatten wir wenig mit GUI oder Grafik im allgemeinen Sinne zu tun. Die meiste Zeit sehen wir Konsolenausgaben in weiß auf schwarz, ein wenig Textausgabe und das war es dann auch schon. 

Wir wollten etwas entwickeln mit dem wir im späteren Leben höchstwahrscheinlich nur noch als Anwender zu tun haben. Ein Spiel. 

Viele Informatik Studenten träumen oder haben davon geträumt ein Spieleentwickler zu werden. Doch meistens wird daraus nichts. Deshalb haben wir uns gedacht bevor wir ins wirkliche Berufsleben einsteigen, wollen wir einmal ein eigenes Spiel entwickeln und haben es JOSIE getauft.

\section{Vorgehen}

Am Anfang war das Nichts.

Eine der schwierigsten Phasen in unserem Projektverlauf war das grobe Design. Wir wollten dass JOSIE jedem aus unserer Gruppe gefällt und jeder seine Ideen einbringen kann.

Nachdem wir in etwa wussten welche Komponenten wir benötigen, haben wir die Aufgabenbereiche auf die Team-Mitglieder wie folgt verteilt.

\begin{itemize}

\item Oleg Geier: Programmierung und Logik

\item Daniel Glück: Grafikdesign und Spieldesign

\item Jonas Kaiser: Spieldesign

\item Tobias Lediger: Storydesign

\item Daniel Mügge: Audiodesign und Grafikdesign

\end{itemize}




\section{Dokumentationsstruktur}



