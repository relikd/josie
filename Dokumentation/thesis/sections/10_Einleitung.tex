\chapter{Einleitung}\label{ch:intro}

\section{Ausgangssituation}\label{sec:1_Ausgangssituation}

In der heutigen Zeit spielt man Videospiele nicht nur auf Computern oder Konsolen, sondern auch auf Mobiltelefonen. Der Markt von Spielen für mobile Geräte ist in den letzten Jahren rapide gewachsen und erfreut sich immer größerer Beliebtheit. Einer Studie von Bitkom \cite{bitkomgaming} aus dem Jahr 2014 zufolge, in der die beliebtesten Spieleplattformn ermittelt wurden, führt das Smartphone bzw. Handy mit 78\% um 9\% gegenüber dem stationären PC und ist somit an erster Stelle. Dieselbe Studie hat sich auch mit den beliebtesten und meist gespielten Spielegenres beschäftigt. Stratiegie- und Denkspiele sind laut Bitcom am beliebtesten, gefolgt von Gelegenheitsspielen, Actionspielen, Social Games, Jump n' Runs und Renn- und Sportspielen.




\section{Motivation}\label{sec:1_Motivation}

Im bisherigen Verlauf unseres Informatik-Studiums hatten wir wenig mit GUI oder Grafik im allgemeinen Sinne zu tun. Die meiste Zeit sehen wir Konsolenausgaben in weiß auf schwarz, ein wenig Textausgabe und das war es dann auch schon. 

Wir wollten etwas entwickeln mit dem wir im späteren Leben höchstwahrscheinlich nur noch als Anwender zu tun haben. Ein Spiel. 

Viele Informatik Studenten träumen oder haben davon geträumt ein Spieleentwickler zu werden. Doch meistens wird daraus nichts. Deshalb haben wir uns gedacht bevor wir ins wirkliche Berufsleben einsteigen, wollen wir einmal ein eigenes Spiel entwickeln und haben es \gamename getauft.
(\sectionauthor{DM})

\section{Vorgehen}\label{sec:1_Vorgehen}

Am Anfang war das Nichts.

Eine der schwierigsten Phasen in unserem Projektverlauf war das grobe Design. Wir wollten dass  \gamename jedem aus unserer Gruppe gefällt und jeder seine Ideen einbringen kann. Deshalb wurden die ersten zwei Wochen des Projekts dem Design gewidmet. 

Nachdem wir wussten welche Komponenten für die Enwicklung benötigt werden, haben wir die Aufgabenbereiche auf die Team-Mitglieder wie folgt verteilt.

\begin{itemize}

\item Oleg Geier: Programmierung und Logik

\item Daniel Glück: Grafikdesign und Spieldesign

\item Jonas Kaiser: Spieldesign und Levelgenerierung

\item Tobias Lediger: Storydesign und Zwischensequenzen

\item Daniel Mügge: Audiodesign und Grafikdesign
	
\end{itemize}



\section{Dokumentationsstruktur}\label{sec:1_Dokumentationsstruktur}
In Kapitel \ref{ch:grundl} werden die Grundlagen erklärt. Diese beinhalten eine Anleitung zur Einrichtung der Entwicklungsumgebung und Einbindung der verwendeten Engine, eine grobe Einführung in die wichtigsten Bestandteile dieser und wie diese in unserem Programm eingesetzt wurden. 
Kapitel \ref{ch:arch} zeigt die Struktur des Spiels und das Zusammenspiel der Klassen und deren Abhängigkeiten mit Hilfe von Diagrammen.
In Kapitel \ref{ch:impl} geben kleine Code--Beispiele einen Einblick über die Implementierung der zur Verfügung stehenden Klassen und Methoden in den Programmcode.
Kapitel \ref{ch:eval} beschäftigt sich mit der Evaluation des Projektes im Hinblick auf das was geschafft und wie gut es umzusetzten wurde, auf den Ablauf des Projektes und die Probleme die während der Entwicklung enstanden, gelöst oder nicht gelöst wurden.
Kapitel \ref{ch:fazit} enthält Ausblicke für die Zukunft des Projektes und ein abschließendes Fazit.


