\chapter{Einleitung}\label{ch:intro}

\section{Ausgangssituation}

\section{Motivation}

Im bisherigen Verlauf unseres Informatik-Studiums hatten wir wenig mit GUI oder Grafik im allgemeinen Sinne zu tun. Die meiste Zeit sehen wir Konsolenausgaben weiß auf schwarz und ein wenig Textausgabe, das wars. 

Wir wollten etwas entwickeln mit dem wir in unserem späteren Leben höchstwahrscheinlich nur noch als Anwender zu tun haben. Ein Spiel. 

Viele Informatik Studenten träumen oder haben davon geträumt ein Spieleentwickler zu werden. Doch meistens wird daraus nichts. Deshalb haben wir uns gedacht bevor wir ins wirkliche Berufsleben einsteigen, wollen wir einmal ein eigenes Spiel entwickeln und haben es JOSIE getauft.

\section{Vorgehen}

Am Anfang war das Nichts.

Eine der schwierigsten Phasen in unserem Projektverlauf war das grobe Design. Wir wollten dass JOSIE jedem aus unserer Gruppe gefällt und jeder seine Ideen einbringen kann.

Nachdem wir in etwa wussten welche Komponenten wir benötigen, haben wir die Aufgabenbereiche auf die Team-Mitglieder verteilt.

\begin{itemize}

\item Oleg Geier: Programmierung

\item Daniel Glück: Grafikdesign

\item Jonas Kaiser: Spieldesign

\item Tobias Lediger: Storydesign

\item Daniel Mügge: Audiodesign

\end{itemize}




\section{Dokumentationsstruktur}



