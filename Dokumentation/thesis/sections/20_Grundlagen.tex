\chapter{Grundlagen}

\section{Framework}

Cocos2d-3.4 Engine
Für unser Projekt haben wir die cocos2d-Engine verwendet, da sie am meisten Möglichkeiten bietet und flexibel ist.

\section{IDE und Plugins}

\section{Was sind Sprites?}
\begin{figure}[h]
  \includegraphics[width=0.1\textwidth]{resources/josie}
  \caption{Das ist Josie}
  \label{fig:josie} 
\end{figure}


\section{Was ist ein Callback?}

In vielen Teilen des Spieles wird \textbf{CC\textunderscore CALLBACK\textunderscore 0()} verwendet. Es handelt sich dabei um eine Referenz, die auf eine Methode einer Instanz verweist. Somit kann man eine Methode oder Funktion dynamisch im Spiel ausführen, wie das Drücken eines Buttons.

\begin{quote}
CC\textunderscore CALLBACK\textunderscore 0(Director::popScene, Director::getInstance());
\end{quote}

Der Director (\textbf{Director::getInstance()}) soll die Funktion \textbf{popScene()} ausführen. Zur Unterscheidung ist es wichtig die Klasse (\textbf{Director::}) zu nennen die ausgeführt werden soll. Denkbar wäre hier eine Kind-Klasse die den selben Methoden Namen aufweist. 
Weitere Parameter der Funktion können Komma getrennt hinten angehängt werden.

\section{Was ist eine TileMap?}