\chapter{Grundlagen}

\section{Framework}

Cocos2d-3.4 Engine

Für unser Projekt haben wir die cocos2d-Engine verwendet, da sie am meisten Möglichkeiten bietet und flexibel ist.

\section{IDE und Plugins}

\section{Szenenprinzip bei Videospielen}

\section{Sprites}
\begin{figure}[h]
  \includegraphics[width=0.1\textwidth]{resources/josie}
  \caption{Das ist Josie}
  \label{fig:josie} 
\end{figure}


\section{Callbackprinzip}

In vielen Teilen des Spieles wird \textbf{CC\textunderscore CALLBACK\textunderscore 0()} verwendet. Es handelt sich dabei um eine Referenz, die auf eine Methode einer Instanz verweist. Somit kann man eine Methode oder Funktion dynamisch im Spiel ausführen, wie das Drücken eines Buttons.

\begin{quote}
CC\textunderscore CALLBACK\textunderscore 0(Director::popScene, Director::getInstance());
\end{quote}

Der Director (\textbf{Director::getInstance()}) soll die Funktion \textbf{popScene()} ausführen. Zur Unterscheidung ist es wichtig die Klasse (\textbf{Director::}) zu nennen die ausgeführt werden soll. Denkbar wäre hier eine Kind-Klasse die den selben Methoden Namen aufweist. 
Weitere Parameter der Funktion können Komma getrennt hinten angehängt werden.

\section{Tilemaps}

\section{Musik und Sound-Effekte}

Musik sowie alle Sounds die in unserem Spiel "'Josie"' zu hören sind wurden selbst geschrieben, aufgenommen und bearbeitet. Dazu gehören:

\begin{itemize}
\item Hintergrundmusik im Hauptmenü, in der Levelauswahl, in den Jump and Run Levels und im Boss Kampf
\item Effektsounds für Sprung--, Schrumpf--, Stop und Schuss--Sounds von Josie, Shop--Sound, Bosstreffer--Sound
\end{itemize}

\subsection{Möglichkeiten der cocos2d-Engine zur Audioverarbeitung}
Cocos2d bietet mit der \textbf{SimpleAudioEngine} eine relative einfache Möglichkeit Audio\-dateien, sei es die Hintergrundmusik oder ein Sound-Effekte, zu laden, abzuspielen, zu pausieren und wieder zu entfernen. Hierzu ein kurzes Beispiel wie man auf einfache Art und Weise eine Audiodatei abspielt.

\begin{quote}
SimpleAudioEngine::getInstance()--\textgreater playBackgroundMusic("'song.mp3"',true);
\end{quote}

Auf die Implementierung und die Verwendung der \textbf{SimpleAudioEngine} innerhalb unseres Codes wird im Kapitel \ref{sec:Audiounit} genauer eingegangen. Vorweg sei gesagt dass wir alle Funktionalitäten welche die \textbf{SimpleAudioEngine} betreffen in eine eigene Klasse \josieclass{Audio\-Unit} ausgelagert haben.

\subsection{Mono-/Stereo-Kanäle und Dateiformate}
Es ist möglich sowohl Mono-- als auch Stereo--Audiodateien zu verwenden. Falls man also möchte dass Sounds zum Beispiel aus bestimmten Richtungen kommen, um dem Spieler ein gewisses Mittendrin--Gefühl zu vermitteln, sollten die Audiodateien stereo sein. Das ist allerdings erst richtig sinnvoll wenn das Spiel mit Kopfhörern oder mit Anschluss an ein Soundsystem gespielt wird. 

In unserem Fall wurden allerdings nur Mono--Audiodateien verwendet da Josie hauptsächlich für möbile Geräte gedacht ist und diese meist nur über einen Lautsprecher verfügen. Und seien wir ehrlich, die meisten Menschen schalten den Ton von Handyspielen aus oder spielen diese ohne Kopfhörer.

Wir haben auschließlich .mp3 verwendet, da dieses Dateiformat in Bezug auf cocos2d von den meisten Geräten unterstützt wird. Ein weiterer Vorteil von .mp3 gegenüber zum Beispiel .wav ist die Dateigröße wobei man auch auf die Bitrate achten muss. 


\subsection{Audiobearbeitungsprogramme}
Auf dem Softwaremarkt gibt es unzählige Audiobearbeitungsprogramme und wenn man sich mit dem Thema Audiobearbeitung noch nie beschäftigt hat, ist es sehr schwer eines zu finden das die nötigen Funktionen liefert um einen gutes Resultat zu erzielen. Zudem kosten die meisten guten Programme viel Geld. Deshalb hier eine kleine Auflistung von kostenlosen und kostenpflichtigen Audiobearbeitungsprogrammen:

\begin{itemize}
 \item Cubase Elements 8 
 \end{itemize} 


