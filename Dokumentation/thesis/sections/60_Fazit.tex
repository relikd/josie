\chapter{Fazit und Ausblick}\label{ch:fazit}

Am Ende eines Projekts sollte man immer Zurückschauen und beurteilen wie dieses lief: Was lief gut? Warum lief es so gut? Wie wurde mit Problemen umgegangen? Was konnte nicht umgesetzt werden?

Diese Fragen sollen nun kurz geklärt werden, um einen Einblick in den Ablauf des Projekts zu ermöglichen.

\sectionJK{Rückblick auf das Projekt}\label{sec:6_review}

Betrachtet man nun - gegen Ende des Projektes – wie dieses verlief so kann man nicht viel am Ablauf kritisieren. Obwohl alle Beteiligten zum ersten Mal mit cocos2dx und sogar dem Erstellen von Spielen Verbindung kamen, gelang es doch schnell, eine Unterteilung in verschiedene Arbeitsbereiche zu erstellen und die Aufgaben fair unter allen Gruppenmitgliedern zu verteilen. Da das Projekt nun fertig ist lässt sich sehen, dass unsere geplanten Arbeitspakete bereits eine gute Schätzung dessen waren, was es tatsächlich zu tun gab; alle Bereiche des Projekts waren bereits unter den Mitgliedern verteilt.

Als es daran ging, die Pläne tatsächlich in die Tat umzusetzen waren es die Gemeinsamen Programmierwochenenden, die das Projekt am schnellsten voranbrachten. Indem wir uns an diesen Tagen ausschließlich auf das Projekt konzentrierten, konnten leicht neue Konzepte entworfen und gleich darauf in die Tat umgesetzt werden. Da wir alle für einen langen Zeitraum vor Ort waren, war es außerdem oft möglich sich gegenseitig schnell Hilfe zu leisten, bzw. fertigen Code eines anderen Gruppenmitglieds kritisch zu betrachten und gemeinsam mit ihm zu überarbeiten. Auf diese Art konnten alle Gruppenmitglieder leicht voneinander lernen und ihre Produktivität so erhöhen.

Selbstverständlich stießen wir im Verlauf des Semesters auch auf Probleme. Allem voran ist dabei das Aussteigen eines Gruppenmitglieds aus dem Projekt zu nennen. Aus persönlichen Gründen entschied sich unser sechstes Gruppenmitglied während der Konzeptionsphase, das Projekt nicht weiter zu bearbeiten. Natürlich entstand dadurch weiterer Aufwand, der  zu Anfang nicht vorhergesehen war: Die Aufgaben mussten neu verteilt werden, die restlichen Gruppenmitglieder mussten sich in neue Bereiche einarbeiten und einige Pläne musste verworfen werden, da es klar war, dass man diese zeitlich nicht mehr umsetzen könnte. Glücklich war in dieser Situation nur, dass dies nicht in einer späteren Phase geschah. Wir konnten die Planung relativ leicht auf die neue Situation umstellen. Andernfalls hätten wir eventuell halb umgesetzte Pläne verwerfen müssen, um sicherzustellen, dass die Arbeit unseres weggefallenen Teammitglieds erfüllt wird.

Zwischenmenschliche Probleme gab es im Projekt glücklicherweise nicht. Obwohl sich die Gruppenmitglieder nicht alle zu Beginn des Projekts kannten entstand schnell eine gewisse Dynamik zwischen allen Beteiligten. Dies erwies sich als besonders nützlich, wann immer die Motivation eines Gruppenmitglieds zu wünschen übrig ließ. Stets genügten ein strenger Blick und eine Aufforderung des Projektleiters, damit die Arbeit fortgesetzt wurde.
 Gegenseitiger Respekt und ein fairer Umgang miteinander waren selbstverständlich. Jeder konnte seine eigene Meinung jederzeit einbringen und auf jeden Vorschlag wurde eingegangen. Es war uns allen wichtig, dass das am Ende entstehende Spiel den Vorstellungen aller Beteiligten entspricht und jeder sich gerne mit dem Projekt identifiziert.


Zusammenfassend kann man das Projekt daher als Erfolg betrachten. Nahezu ohne Ausnahme gelang es uns unsere Pläne in die Tat umzusetzen. Die Features, die wir nicht einfügen konnten, wären zwar sinnvoll, um den Umfang des Spiels zu erweitern, jedoch sinkt ohne sie nicht die Qualität unseres Ergebnisses.  

\sectionDM{Features for the Future}\label{sec:6_Features}

Aus zeitlichen Gründen konnten wir einige Features nicht umsetzen, dazu gehören:

\paragraph{Mehr Level}
Die momentane Levelstruktur 1.1--1.2--1.3--Boss könnte in die vertikale Ebene erweitert werden. In Zukunft wären mehrere Levelebenen (2.1--2.2--2.3--Boss, etc) wünschenswert. Durch eine höhere Anzahl von Levels ist auch eine interessante Story mit Cutscenes und neuen Umgebungen besser umsetzbar. 

\paragraph{Erweiterung der Levels}
Eine Bereicherung des Spiels wären außerdem neue Mechaniken und Gegner im Jump and Run Level. Mit Mechaniken sind zum Beispiel Plattformen gemeint auf denen man stehen bleiben muss um von Punkt A nach Punkt B zu gelangen. Zusätzlich dazu könnte man Gegner einführen, bei deren Berührung der Spieler stirbt. Sozusagen sich bewegende, tödliche Hindernisse.

\paragraph{Mehr Endgegner}
Neue Bosse mit neuen Angriffspatterns und Bewegungsabläufen, neuen Designs und interessanteren Mechaniken wären eine weitere große Ergänzung die das Spiel noch besser machen würde. Hier war es angedacht die BossLevel Klasse so umzuschreiben, damit man neue Gegner durch Subklassen erstellen kann und nur noch die Grafiken und Angriffspattern implementiert muss.

\paragraph{Neue Aufwertungsmöglichkeiten}
Die bestehenden Aufwertungen im Shop könnte man beispielsweise mit Element-Projektilen erweitern, die über verschiedene Schadensarten verfügen und in Abhängigkeit zum Element des Boss Gegners mehr oder weniger Schaden verursachen. Hierfür wären auch neue Projektil-Grafiken erforderlich was das Spiel fürs Auge interessanter machen würde.

\paragraph{Lade Bildschirm}
Das Initialisieren eines Levels soll im Hintergrund geschen und dem Anwender ein Lade Bildschirm angezeigt. Das steigert die gefühlte Ausführungszeit und dem Benutzer wird immer etwas geboten.

\paragraph{Double-Jump-Gliding}
Die Erweiterung der Sprungfunktion könnte nach einem weiteren Klick in der Luft dazu führen, dass Josie ein kleines Stück in der Luft gleitet. Das hätte zur Folge dass man auch Levelabschnitte mit größeren Sprungabständen einbauen könnte.

\paragraph{Neue Kampfmodi}
Damit ist gemeint, dass Josie sich in einen Hubschrauber verwandeln und den Boss von oben bekämpfen kann oder in einen Mech/Roboter, der anstatt einer Links-Rechts-Bewegung lediglich einen Sprung ausführt und den Boss von der Seite bekämpft.

\paragraph{Speichern und Laden des Random Levels}
Derzeit kann derselbe auotmatisch generierte Level  nur wiederholt werden, solange der Spieler ihn nicht vollständig verlässt. Es wäre daher nützlich eine Funktion zu haben, welche die Tilemap des Levels als TMX Datei abspeichert. Alle diese Dateiene könnten dann später in einem extra Menü ausgewählt werden um die LEvel erneut zu spielen, die dem Spieler am besten gefallen haben. 
