\chapter{Fazit und Ausblick}\label{ch:fazit}

Kritisch betrachtet lässt sich sagen, dass unser Spiel ein vollständig funktionierendes 2D-Sidescroll-Game darstellt. Mit komplett individuellem Audio, Animations und Grafikdesign sowie eigener Steuerung. 

\sectionDM{Features for the Future}\label{sec:6_Features}

Aus zeitlichen Gründen konnten wir einige Features nicht umsetzen, dazu gehören:

\paragraph{Mehr Level}
Die momentane Levelstruktur 1.1--1.2--1.3--Boss könnte in die vertikale Ebene erweitert werden. In Zukunft wären mehrere Levelebenen (2.1--2.2--2.3--Boss, etc) wünschenswert. Durch eine höhere Anzahl von Levels ist auch eine interessante Story mit Cutscenes und neuen Umgebungen besser umsetzbar. 

\paragraph{Erweiterung der Levels}
Eine Bereicherung des Spiels wären außerdem neue Mechaniken und Gegner im Jump and Run Level. Mit Mechaniken sind zum Beispiel Plattformen gemeint auf denen man stehen bleiben muss um von Punkt A nach Punkt B zu gelangen. Zusätzlich dazu könnte man Gegner einführen, bei deren Berührung der Spieler stirbt. Sozusagen sich bewegende, tödliche Hindernisse.

\paragraph{Mehr Endgegner}
Neue Bosse mit neuen Angriffspatterns und Bewegungsabläufen, neuen Designs und interessanteren Mechaniken wären eine weitere große Ergänzung die das Spiel noch besser machen würde. Hier war es angedacht die BossLevel Klasse so umzuschreiben, damit man neue Gegner durch Subklassen erstellen kann und nur noch die Grafiken und Angriffspattern implementiert muss.

\paragraph{Neue Aufwertungsmöglichkeiten}
Die bestehenden Aufwertungen im Shop könnte man beispielsweise mit Element-Projektilen erweitern, die über verschiedene Schadensarten verfügen und in Abhängigkeit zum Element des Boss Gegners mehr oder weniger Schaden verursachen. Hierfür wären auch neue Projektil-Grafiken erforderlich was das Spiel fürs Auge interessanter machen würde.

\paragraph{Lade Bildschirm}
Das Initialisieren eines Levels soll im Hintergrund geschen und dem Anwender ein Lade Bildschirm angezeigt. Das steigert die gefühlte Ausführungszeit und dem Benutzer wird immer etwas geboten.

\paragraph{Double-Jump-Gliding}
Die Erweiterung der Sprungfunktion könnte nach einem weiteren Klick in der Luft dazu führen, dass Josie ein kleines Stück in der Luft gleitet. Das hätte zur Folge dass man auch Levelabschnitte mit größeren Sprungabständen einbauen könnte.

\paragraph{Neue Kampfmodi}
Damit ist gemeint, dass Josie sich in einen Hubschrauber verwandeln und den Boss von oben bekämpfen kann oder in einen Mech/Roboter, der anstatt einer Links-Rechts-Bewegung lediglich einen Sprung ausführt und den Boss von der Seite bekämpft.
